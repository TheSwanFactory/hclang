\documentclass[preprint]{{sigplanconf}}
% generated by Madoko, version 1.1.1
%mdk-data-line={1}


\usepackage[heading-base={2},section-num={False},bib-label={hide},fontspec={True}]{madoko2}
\usepackage{amsmath}


\begin{document}



\begin{mdpre}%mdk
\noindent~;~.hello~"World"\\
~\#~world\\
~\#~comment\\
~\\
~%mdk
\end{mdpre}
\mdxtitleblockstart{}
\mdxtitle{Homoiconic C}%mdk

\mdxsubtitle{Programming without a Language}%mdk
\mdtitleauthorrunning{}{}\mdxtitleblockend%mdk

\begin{abstract}%mdk

\noindent\mdbr
Modern programming is dominated by languages derived from C{}[\mdcite{clang}{1}] and
the UNIX shell{}[\mdcite{unixshell}{12}], characterized by complicated syntax and
sophisticated parsers. Minimalist languages such as Forth{}[\mdcite{forth}{3}],
Scheme{}[\mdcite{scheme}{8}] and Self{}[\mdcite{self}{9}] are no longer taken seriously\mdfootnote{1}{\noindent The only minimalist language still in active use appears to be
Lua{}[\mdcite{lua}{7}], though even that is relegated to niche applications.%mdk
\label{fn-lua-fn}%mdk%mdk
},
presumably because they are felt to lack sufficiently readable \emph{syntax}
and powerful enough \emph{semantics}.%mdk

In this paper we introduce a novel data format called \textbf{Homoiconic C}
(\textquotedblleft{}HC\textquotedblright{}) that we claim expresses a safe superset of C-like computations,
yet is even easier to read than existing scripting languages. The key is
a novel data structure and computational model we call \emph{Frames}, designed
to replace and improve upon the s-expressions and metacircular evaluation
model of Lisp{}[\mdcite{lisp}{6}]. Frames act as a universal monad\mdfootnote{2}{\noindent Not just the mathematical term for an object closed under all
operations, but Liebniz's original idea of a single universal
building-block{}[\mdcite{liebniz}{5}].%mdk
\label{fn-monad}%mdk%mdk
}, with the
result that Homoiconic C is simply conventions for creating and
evaluating Frames.%mdk

In this paper we will describe the complete syntax of HC, including three
types of lists, seven atoms, and the seven predefined operators; as a
modeless data format, there are no keywords, reserved words, or even any
grammar. Next we discuss the robust data protection semantics inspired by
BitC{}[\mdcite{shapiro:effecttyping}{10}], which allow us the efficiency of C's memory
model without any of the downsides. Finally, we provide examples of how
these deceptively simple primitives allow us to trivially implement HTML,
Object-Oriented programming, and other higher-level abstractions.%mdk

We are currently developing the first implementation of HC as a
TypeScript{}[\mdcite{typescript}{11}] interpreter. It can be found at
\href{http://github.com/TheSwanFactory/hclang/}{{\ttfamily http://\hspace{0pt}github.\hspace{0pt}com/\hspace{0pt}TheSwanFactory/\hspace{0pt}hclang/\hspace{0pt}}} under an MIT Open Source
license.%mdk
%mdk
\end{abstract}%mdk

\section{1.\hspace*{0.5em}Introduction}\label{sec-intro}%mdk%mdk

\noindent Homoiconic C is a data format for computation. It is designed to be a
simple yet powerful alternative to both traditional programming languages
and existing data formats.%mdk

\subsection{1.1.\hspace*{0.5em}A Brief History of Programming Languages}\label{sec-a-brief-history-of-programming-languages}%mdk%mdk

\noindent Early high-level languages (e.g., FORTRAN{}[\mdcite{cite}{2}], COBOL{}[\mdcite{cite}{2}],
BASIC{}[\mdcite{cite}{2}], ALGOL{}[\mdcite{cite}{2}]) were strongly influenced by mathematics and artificial
intelligence{}[\mdcite{cite}{2}], leading to sophisticated precedence rules and
complicated syntax. The primary alternative was Lisp[Stub~{}[\mdcite{lisp}{6}];], defined \textendash{}
at least in theory \textendash{} by a single data structure (the list) and a single
rule (metacircular evaluation).%mdk

Lisp was not as efficient as, say, Fortran, but it was much more elegant
and powerful. The common assumption was that eventually computers would
become powerful enough that Lisp's inefficiencies would no longer matter.
Sadly, the failure of hardware-optimized Lisp machines{}[\mdcite{cite}{2}] marked the death
of that dream.%mdk

\subsection{1.2.\hspace*{0.5em}Then Came C}\label{sec-then-came-c}%mdk%mdk

\noindent Into that void stepped the C programming language{}[\mdcite{clang}{1}]. C was derived from ALGOL,
but borrowed (some would say butchered{}[\mdcite{cite}{2}]) several powerful ideas
from Lisp, such as macros{}[\mdcite{cite}{2}] and function pointers{}[\mdcite{cite}{2}]. In
addition, the C memory model{}[\mdcite{cite}{2}] mapped well onto modern processor
architectures, enabling remarkably good performance. It also benefited from:%mdk

\begin{itemize}[noitemsep,topsep=\mdcompacttopsep]%mdk

\item A relatively simple and easy to read syntax%mdk

\item Association with the UNIX operating system%mdk

\item Self-sustaining{}[\mdcite{cite}{2}], open source compilers%mdk
%mdk
\end{itemize}%mdk

\noindent This led to C (along with its direct descendants, C++{}[\mdcite{cite}{2}] and
Objective-C{}[\mdcite{cite}{2}]) completely dominating the field for systems
programming; a situation which continues today, almost fifty years
later. This led to the C syntax (with some borrowing from its sister language, the UNIX shell) 
becoming the baseline for virtually every mass-market language in use today{}[\mdcite{cite}{2}].%mdk

\subsection{1.3.\hspace*{0.5em}The Downside of C}\label{sec-the-downside-of-c}%mdk%mdk

\noindent This success was not entirely a good thing. C's memory model and type
system optimized for efficiency at the cost of safety, leading to
innumerable programming errors and security holes that still plague us
today~{}[\mdcite{cite}{2}]. C's syntax and macros \textendash{}  a breath of fresh air
when they first came out \textendash{} feel clunky and dated in a world of scripting
languages and functional programming, to the point where it is rarely
taught to beginning programmers anymore.%mdk

In the last decade, we have seen renewed innovation in systems
programming, leading to languages such as Rust{}[\mdcite{cite}{2}], Go{}[\mdcite{cite}{2}], and
Swift~{}[\mdcite{cite}{2}]. These are enough better and different than C that they
have carved out their own niches (especially with the help of powerful
patrons). However, they don't seem to have the factor-of-ten improvement
necessary to displace or replace C, the way C did to assembly language.%mdk

\subsection{1.4.\hspace*{0.5em}Framing A New Hope}\label{sec-frames}%mdk%mdk

\noindent Homoiconic C aspires to reach that goal by marrying the best aspects of
C's syntax and memory model with the values of rigor and simplicity that
characterized Lisp. The central innovation is a novel data structure we
call a \textbf{frame}. Frames are simultaneously:%mdk

\begin{itemize}[noitemsep,topsep=\mdcompacttopsep]%mdk

\item Callable (like functions)%mdk

\item Enumerable (like arrays)%mdk

\item Associative (like structs)%mdk

\item Inherited (like classes)%mdk
%mdk
\end{itemize}%mdk

\noindent Our premise is that the multitude of structures found in programming
languages boil down to just these four attributes. By encoding
them in a single object, we hope to dramatically simplify both the structure
of the language and the process of programming.%mdk

HC also incorporates:%mdk

\begin{mddefinitions}%mdk

\mddefterm{\noindent{\bfseries Homoiconicity, from Lisp}}%mdk

\begin{mdbmarginx}{}{}{}{1.5em}%mdk
\begin{mddefdata}%mdk
Frames are both code and data, making it trivial to manipulate code via higher-level abstractions.
%mdk
\end{mddefdata}%mdk
\end{mdbmarginx}%mdk

\mddefterm{\noindent{\bfseries Dataflow, from the UNIX Shell}}%mdk

\begin{mdbmarginx}{}{}{}{1.5em}%mdk
\begin{mddefdata}%mdk
Iteration and Input/Output are designed for simple pipeline-style composition.
%mdk
\end{mddefdata}%mdk
\end{mdbmarginx}%mdk

\mddefterm{\noindent{\bfseries Effect Typing, from BitC{}[\mdcite{cite}{2}] }}%mdk

\begin{mdbmarginx}{}{}{}{1.5em}%mdk
\begin{mddefdata}%mdk
By carefully annotating function boundaries for both \emph{constancy} and \emph{immutability}, we can safely
alias data structures to get C-like performance while preserving the security of managed code.%mdk
\end{mddefdata}%mdk
\end{mdbmarginx}%mdk
%mdk
\end{mddefinitions}%mdk

\subsection{1.5.\hspace*{0.5em}About This Document}\label{sec-about-this-document}%mdk%mdk

\noindent In the remainder of this document, we will cover:%mdk

\begin{itemize}[noitemsep,topsep=\mdcompacttopsep]%mdk

\item The syntax, operators, and data protection rules that define the language%mdk

\item Examples of using Homoiconic C to model
object-oriented programming and the full stack of web technologies%mdk

\item The current status and possible future directions%mdk

\item Conclusions and comparisons to related work%mdk
%mdk
\end{itemize}%mdk

\section{2.\hspace*{0.5em}Language Definition}\label{sec-language-definition}%mdk%mdk

\subsection{2.1.\hspace*{0.5em}Syntax}\label{sec-syntax}%mdk%mdk

\noindent The core principle of Homoiconic C syntax is \emph{isomorphism}: every
syntactic construct refers to exactly one semantic concept. Put another
way, it is modeless: characters don't mean different things in different
places. This enables vastly simpler parsers and a much shallower learning
curve.%mdk

Like traditional data formats, HC is build around \emph{literals} and
the \emph{terminals} that separate and organize them. HC simply extends that
with \emph{identifiers}, and a single grammatic construct, the \emph{expression}.%mdk

\subsubsection{2.1.1.\hspace*{0.5em}Data Format}\label{sec-data-format}%mdk%mdk

\noindent Homoiconic C is intended to be a universal format for data as well as code.
The pure data constructs are found in Table~\mdref{sec-table-data}{\mdcaptionlabel{1}}.%mdk

\begin{table}[tbp]%mdk
\begin{mdcenter}%mdk
\begin{mdtabular}{5}{\dimeval{(\linewidth)/5}}{1ex}%mdk
\begin{tabular}{lllll}\midrule
\multicolumn{5}{|c|}{{\bfseries Literals}}\\
\multicolumn{2}{|c}{{\mdcellcolor{floralwhite}}{\bfseries Bit}}&\multicolumn{2}{|c}{{\mdcellcolor{floralwhite}}{\bfseries Number}}&\multicolumn{1}{|c|}{{\mdcellcolor{floralwhite}}{\bfseries Quote}}\\

\midrule
\multicolumn{1}{|l}{{\mdcellcolor{gainsboro}}\textless{}\textgreater{}}&{\mdcellcolor{gainsboro}}\emph{all (true)}&\multicolumn{1}{|l}{{\mdcellcolor{gainsboro}}0b010}&{\mdcellcolor{gainsboro}}\emph{binary}&\multicolumn{1}{|l|}{{\mdcellcolor{gainsboro}}“\emph{string}”}\\
\multicolumn{1}{|l}{{\mdcellcolor{floralwhite}}()}&{\mdcellcolor{floralwhite}}\emph{nil (false)}&\multicolumn{1}{|l}{{\mdcellcolor{floralwhite}}0o1777}&{\mdcellcolor{floralwhite}}\emph{octal}&\multicolumn{1}{|l|}{{\mdcellcolor{floralwhite}}\# \emph{comment} \#}\\
\multicolumn{2}{|l}{{\mdcellcolor{gainsboro}}}&\multicolumn{1}{|l}{{\mdcellcolor{gainsboro}}1234}&{\mdcellcolor{gainsboro}}\emph{decimal}&\multicolumn{1}{|l|}{{\mdcellcolor{gainsboro}}\textbackslash{}\emph{length}\textbackslash{}\emph{blob}}\\
\multicolumn{2}{|l}{{\mdcellcolor{floralwhite}}}&\multicolumn{1}{|l}{{\mdcellcolor{floralwhite}}0xCAFE}&{\mdcellcolor{floralwhite}}\emph{hexadecimal}&\multicolumn{1}{|l|}{{\mdcellcolor{floralwhite}}\%\emph{date\_time}\%}\\
\multicolumn{2}{|l}{{\mdcellcolor{gainsboro}}}&\multicolumn{1}{|l}{{\mdcellcolor{gainsboro}}0@Base64}&{\mdcellcolor{gainsboro}}\emph{Base64}&\multicolumn{1}{|l|}{{\mdcellcolor{gainsboro}}}\\
\midrule
\end{tabular}\end{mdtabular}
\begin{mdtabular}{6}{\dimeval{(\linewidth)/6}}{1ex}%mdk
\begin{tabular}{llllll}\midrule
\multicolumn{6}{|c|}{{\bfseries Terminals}}\\
\multicolumn{2}{|c}{{\mdcellcolor{floralwhite}}{\bfseries Separators}}&\multicolumn{2}{|c}{{\mdcellcolor{floralwhite}}{\bfseries Aggregates}}&\multicolumn{2}{|c|}{{\mdcellcolor{floralwhite}}{\bfseries Whitespace}}\\

\midrule
\multicolumn{1}{|l}{{\mdcellcolor{gainsboro}},}&{\mdcellcolor{gainsboro}}\emph{end expression}&\multicolumn{1}{|l}{{\mdcellcolor{gainsboro}}(}&{\mdcellcolor{gainsboro}}\emph{begin group}&\multicolumn{1}{|l}{{\mdcellcolor{gainsboro}}space}&\multicolumn{1}{l|}{{\mdcellcolor{gainsboro}}\emph{separate tokens}}\\
\multicolumn{1}{|l}{{\mdcellcolor{floralwhite}};}&{\mdcellcolor{floralwhite}}\emph{end statement}&\multicolumn{1}{|l}{{\mdcellcolor{floralwhite}})}&{\mdcellcolor{floralwhite}}\emph{end group}&\multicolumn{1}{|l}{{\mdcellcolor{floralwhite}}newline}&\multicolumn{1}{l|}{{\mdcellcolor{floralwhite}}\emph{end group}}\\
\multicolumn{2}{|l}{{\mdcellcolor{gainsboro}}}&\multicolumn{1}{|l}{{\mdcellcolor{gainsboro}}]}&{\mdcellcolor{gainsboro}}\emph{end array}&\multicolumn{2}{|l|}{{\mdcellcolor{gainsboro}}}\\
\multicolumn{2}{|l}{{\mdcellcolor{floralwhite}}}&\multicolumn{1}{|l}{{\mdcellcolor{floralwhite}}{}[}&{\mdcellcolor{floralwhite}}\emph{begin closure}&\multicolumn{2}{|l|}{{\mdcellcolor{floralwhite}}}\\
\multicolumn{2}{|l}{{\mdcellcolor{gainsboro}}}&\multicolumn{1}{|l}{{\mdcellcolor{gainsboro}}]}&{\mdcellcolor{gainsboro}}\emph{end closure}&\multicolumn{2}{|l|}{{\mdcellcolor{gainsboro}}}\\
\midrule
\end{tabular}\end{mdtabular}

\mdhr{}%mdk

\noindent\mdcaption{\textbf{Table~\mdcaptionlabel{1}.}~\mdcaptiontext{Data Format Specification}}%mdk
%mdk
\end{mdcenter}\label{sec-table-data}%mdk
%mdk
\end{table}%mdk

To be precise, '\textless{}\textgreater{}' is the total type and \textquoteleft{}()\textquoteright{} is the empty expression, but since they evaluate to
themselves they can be considered literals.%mdk

\subsubsection{2.1.2.\hspace*{0.5em}Identifiers}\label{sec-identifiers}%mdk%mdk

\begin{table}[tbp]%mdk
\begin{mdcenter}%mdk
\begin{mdtabular}{4}{\dimeval{(\linewidth)/4}}{1ex}%mdk
\begin{tabular}{llll}\midrule
\multicolumn{4}{|c|}{{\bfseries Identifiers}}\\
\multicolumn{1}{|c}{{\mdcellcolor{floralwhite}}{\bfseries Assignment}}&\multicolumn{1}{|c}{{\mdcellcolor{floralwhite}}{\bfseries Constancy}}&\multicolumn{1}{|c}{{\mdcellcolor{floralwhite}}{\bfseries Privacy}}&\multicolumn{1}{|c|}{{\mdcellcolor{floralwhite}}{\bfseries Effect}}\\

\midrule
\multicolumn{1}{|l}{{\mdcellcolor{gainsboro}}\emph{value}}&\multicolumn{1}{|l}{{\mdcellcolor{gainsboro}}\emph{variable}}&\multicolumn{1}{|l}{{\mdcellcolor{gainsboro}}\emph{public}}&\multicolumn{1}{|l|}{{\mdcellcolor{gainsboro}}\emph{immutable}}\\
\multicolumn{1}{|l}{{\mdcellcolor{floralwhite}}.\emph{name}}&\multicolumn{1}{|l}{{\mdcellcolor{floralwhite}}\emph{Constant}}&\multicolumn{1}{|l}{{\mdcellcolor{floralwhite}}\_\emph{protected}}&\multicolumn{1}{|l|}{{\mdcellcolor{floralwhite}}\emph{mutable}\_}\\
\multicolumn{1}{|l}{{\mdcellcolor{gainsboro}}}&\multicolumn{1}{|l}{{\mdcellcolor{gainsboro}}}&\multicolumn{1}{|l}{{\mdcellcolor{gainsboro}}\_\_\emph{private}}&\multicolumn{1}{|l|}{{\mdcellcolor{gainsboro}}\emph{mutating}:}\\
\midrule
\end{tabular}\end{mdtabular}
\begin{mdtabular}{1}{\dimeval{(\linewidth)/1}}{1ex}%mdk
\begin{tabular}{l}\midrule
\multicolumn{1}{|c|}{{\bfseries Special}}\\

\midrule
\multicolumn{1}{|l|}{{\mdcellcolor{gainsboro}}@\emph{reference}}\\
\multicolumn{1}{|l|}{{\mdcellcolor{floralwhite}}\$\emph{control}}\\
\midrule
\end{tabular}\end{mdtabular}

\mdhr{}%mdk

\noindent\mdcaption{\textbf{Table~\mdcaptionlabel{2}.}~\mdcaptiontext{Identifiers}}%mdk
%mdk
\end{mdcenter}\label{sec-table-id}%mdk
%mdk
\end{table}%mdk

\subsubsection{2.1.3.\hspace*{0.5em}Expressions}\label{sec-expressions}%mdk%mdk

\noindent Expressions are simply a list of frames evaluated within the current
context then applied left to right (left fold~{}[\mdcite{leftfold}{4}]). We refer to
this as the \emph{elliptical evaluator}, in homage to Lisp's metacircular
evaluator.%mdk
\begin{mdpre}%mdk
\noindent\preindent{2}frame0~frame1~frame2~...%mdk
\end{mdpre}\noindent This is equivalent to the explicit grouping:
\begin{mdpre}%mdk
\noindent\preindent{2}((frame0~frame1)~frame2)%mdk
\end{mdpre}\noindent The only variation on this is deferred evaluation within braces:
\begin{mdpre}%mdk
\noindent\preindent{2}\{~frame0~frame1~frame2~\}%mdk
\end{mdpre}\noindent That's all there is. There are no special forms, keywords, precedence rules, or
other grammatic constructs. While we realize those may provide some
efficiency gains for experienced programmers, we do not believe that is 
worth of cost of complicating the implementation and increasing the
learning curve for new programmers.

\paragraph{Assignment}\label{sec-assignment}%mdk%mdk

\noindent In particular, assignment is just a simple expression applying a \emph{value} to a \emph{name}:%mdk
\begin{mdpre}%mdk
\noindent\preindent{2}.key~value;%mdk
\end{mdpre}\noindent This avoids the subtle and confusing distinction between differing
\textquotedblleft{}left-hand-side\textquotedblright{} and \textquotedblleft{}right-hand-side\textquotedblright{} interpretations of an identical symbol \emph{x}.
It also save us from overloading \textquoteleft{}=\textquoteright{} to mean both assignment and a test for equality.

\paragraph{Properties}\label{sec-properties}%mdk%mdk

\noindent Names also provide an elegant way of manipulating data structures, still in the context of simple expressions:%mdk
\begin{mdpre}%mdk
\noindent\preindent{2}structure~.property;~\#~gets~a~property\\
\preindent{2}.structure.property~value~\#~sets~a~property~%mdk
\end{mdpre}\noindent The space between \emph{structure} and \emph{property} is not necessary, but we include it to emphasize the fact
that this is just an ordinary expression, not a special syntax.

Note that due to the scoping rules, the value \emph{structure} can be either
local or in any parent scope, but the name \emph{.structure} is always local.  If we wish to 
re-assign that parent value, we use an \emph{@reference}:%mdk
\begin{mdpre}%mdk
\noindent\preindent{2}@structure.property~value~\#~sets~a~property~in~structure's~current~scope~~%mdk
\end{mdpre}
\subsection{2.2.\hspace*{0.5em}Operators}\label{sec-operators}%mdk%mdk

\noindent To provide the functionality%mdk

\begin{table}[tbp]%mdk
\begin{mdcenter}%mdk
\begin{mdtabular}{4}{\dimeval{(\linewidth)/4}}{1ex}%mdk
\begin{tabular}{llll}\midrule
\multicolumn{4}{|c|}{{\bfseries Content}}\\
\multicolumn{1}{|c}{{\mdcellcolor{floralwhite}}{\bfseries Role}}&\multicolumn{1}{|c}{{\mdcellcolor{floralwhite}}{\bfseries Data}}&\multicolumn{1}{|c}{{\mdcellcolor{floralwhite}}{\bfseries Metadata}}&\multicolumn{1}{|c|}{{\mdcellcolor{floralwhite}}{\bfseries Both}}\\

\midrule\morecmidrules
\cmidrule{4-4}\morecmidrules
\cmidrule{4-4}
\multicolumn{1}{|l}{{\mdcellcolor{gainsboro}}Map}&\multicolumn{1}{|l}{{\mdcellcolor{gainsboro}}\textbar{}}&\multicolumn{1}{|l}{{\mdcellcolor{gainsboro}}\textbar{}\textbar{}}&\multicolumn{1}{|l|}{{\mdcellcolor{gainsboro}}\textbar{}\textbar{}\textbar{}}\\
\multicolumn{1}{|l}{{\mdcellcolor{floralwhite}}Fold}&\multicolumn{1}{|l}{{\mdcellcolor{floralwhite}}\&}&\multicolumn{1}{|l}{{\mdcellcolor{floralwhite}}\&\&}&\multicolumn{1}{|l|}{{\mdcellcolor{floralwhite}}\&\&\&}\\
\midrule
\end{tabular}\end{mdtabular}
\begin{mdtabular}{6}{\dimeval{(\linewidth)/6}}{1ex}%mdk
\begin{tabular}{llllll}\midrule
\multicolumn{5}{|c}{{\bfseries Structure}}&\multicolumn{1}{c|}{{\bfseries}}\\
\multicolumn{2}{|c}{{\mdcellcolor{floralwhite}}{\bfseries Conditional}}&\multicolumn{2}{|c}{{\mdcellcolor{floralwhite}}{\bfseries Type}}&\multicolumn{1}{|c}{{\mdcellcolor{floralwhite}}{\bfseries Bind}}&\multicolumn{1}{c|}{{\mdcellcolor{floralwhite}}{\bfseries}}\\

\midrule
\multicolumn{1}{|l}{{\mdcellcolor{gainsboro}}?}&{\mdcellcolor{gainsboro}}\emph{if}&\multicolumn{1}{|l}{{\mdcellcolor{gainsboro}}\textless{}}&{\mdcellcolor{gainsboro}}\emph{begin type}&\multicolumn{2}{|l|}{{\mdcellcolor{gainsboro}}“\emph{string}”}\\
\multicolumn{1}{|l}{{\mdcellcolor{floralwhite}}:}&{\mdcellcolor{floralwhite}}\emph{else}&\multicolumn{1}{|l}{{\mdcellcolor{floralwhite}}\textgreater{}}&{\mdcellcolor{floralwhite}}\emph{end type}&\multicolumn{2}{|l|}{{\mdcellcolor{floralwhite}}\# \emph{comment} \#}\\
\midrule
\end{tabular}\end{mdtabular}

\mdhr{}%mdk

\noindent\mdcaption{\textbf{Table~\mdcaptionlabel{3}.}~\mdcaptiontext{Pre-Defined Operators}}%mdk
%mdk
\end{mdcenter}\label{sec-table-op}%mdk
%mdk
\end{table}%mdk

Note the type is just a pair of operators, not a syntactic construct,
since it does not change the evaluation rules.%mdk

\section{3.\hspace*{0.5em}Data Protection}\label{sec-data-protection}%mdk%mdk

\subsection{3.1.\hspace*{0.5em}Encapsulation}\label{sec-encapsulation}%mdk%mdk

\subsection{3.2.\hspace*{0.5em}Effect Typing}\label{sec-effect-typing}%mdk%mdk

\section{4.\hspace*{0.5em}Applications}\label{sec-applications}%mdk%mdk

\subsection{4.1.\hspace*{0.5em}Object-Orientation}\label{sec-object-orientation}%mdk%mdk

\noindent Let's program some Javascript:%mdk

\subsection{4.2.\hspace*{0.5em}Web Technologies}\label{sec-web-technologies}%mdk%mdk

\section{5.\hspace*{0.5em}Next Steps}\label{sec-next-steps}%mdk%mdk

\subsection{5.1.\hspace*{0.5em}Implementation Status}\label{sec-implementation-status}%mdk%mdk

\subsection{5.2.\hspace*{0.5em}Future Directions}\label{sec-future-directions}%mdk%mdk

\subsection{5.3.\hspace*{0.5em}Potential Implications}\label{sec-potential-implications}%mdk%mdk

\section{6.\hspace*{0.5em}Related Work}\label{sec-related-work}%mdk%mdk

\noindent\textbf{Note}.
The syntax highlighting works in the PDF output too.%mdk%mdk

\section{7.\hspace*{0.5em}Conclusion}\label{sec-conclusion}%mdk%mdk

\noindent Really fun to write Markdown :-)%mdk

\mdsetrefname{References}%mdk
{\mdbibindent{0}%mdk
\begin{thebibliography}{12}%mdk
\label{sec-bibliography}%mdk

\bibitem{clang}\mdbibitemlabel{}Stub.
\newblock  \emph{Stub}.
\newblock  Stub, 1900a.\label{clang}%mdk%mdk

\bibitem{cite}\mdbibitemlabel{}Stub.
\newblock  \emph{Stub}.
\newblock  Stub, 1900b.\label{cite}%mdk%mdk

\bibitem{forth}\mdbibitemlabel{}Stub.
\newblock  \emph{Stub}.
\newblock  Stub, 1900c.\label{forth}%mdk%mdk

\bibitem{leftfold}\mdbibitemlabel{}Stub.
\newblock  Stub.
\newblock  \emph{Stub}, 1900d.\label{leftfold}%mdk%mdk

\bibitem{liebniz}\mdbibitemlabel{}Stub.
\newblock  Stub.
\newblock  \emph{Stub}, 1900e.\label{liebniz}%mdk%mdk

\bibitem{lisp}\mdbibitemlabel{}Stub.
\newblock  Stub.
\newblock  \emph{Stub}, 1900f.\label{lisp}%mdk%mdk

\bibitem{lua}\mdbibitemlabel{}Stub.
\newblock  Stub.
\newblock  \emph{Stub}, 1900g.\label{lua}%mdk%mdk

\bibitem{scheme}\mdbibitemlabel{}Stub.
\newblock  \emph{Stub}.
\newblock  Stub, 1900h.\label{scheme}%mdk%mdk

\bibitem{self}\mdbibitemlabel{}Stub.
\newblock  Stub.
\newblock  \emph{Stub}, 1900i.\label{self}%mdk%mdk

\bibitem{shapiro:effecttyping}\mdbibitemlabel{}Stub.
\newblock  Stub.
\newblock  \emph{Stub}, 1900j.\label{shapiro:effecttyping}%mdk%mdk

\bibitem{typescript}\mdbibitemlabel{}Stub.
\newblock  Stub.
\newblock  \emph{Stub}, 1900k.\label{typescript}%mdk%mdk

\bibitem{unixshell}\mdbibitemlabel{}Stub.
\newblock  \emph{Stub}.
\newblock  Stub, 1900l.\label{unixshell}%mdk%mdk
\par%mdk
\end{thebibliography}}%mdk%mdk

\begin{mdbmargintb}{4em}{}%mdk
\begin{mdflushright}%mdk
{\tiny Created with~\href{https://www.madoko.net}{Madoko.net}.}%mdk
\end{mdflushright}%mdk
\end{mdbmargintb}%mdk%mdk


\end{document}
