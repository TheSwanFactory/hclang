\documentclass[preprint]{{sigplanconf}}
% generated by Madoko, version 1.1.0
%mdk-data-line={1}


\usepackage[heading-base={2},section-num={False},bib-label={hide},fontspec={True}]{madoko2}
\usepackage{amsmath}


\begin{document}



\mdxtitleblockstart{}
\mdxtitle{Homoiconic C}%mdk

\mdxsubtitle{Programming without a Language}%mdk
\mdtitleauthorrunning{}{}\mdxtitleblockend%mdk

\begin{abstract}%mdk

\noindent\mdbr
Modern programming is dominated by languages derived from C{}[\mdcite{clang}{1}] and
the UNIX shell{}[\mdcite{unixshell}{10}], characterized by complicated syntax and
sophisticated parsers. Minimalist languages such as Forth{}[\mdcite{forth}{2}],
Scheme{}[\mdcite{scheme}{6}] and Self{}[\mdcite{self}{7}] are no longer taken seriously\mdfootnote{1}{\noindent The only minimalist language still in active use appears to be
Lua{}[\mdcite{lua}{5}], though even that is relegated to niche applications.%mdk
\label{fn-lua-fn}%mdk%mdk
},
presumably because they are felt to lack sufficiently readable \emph{syntax}
and powerful enough \emph{semantics}.%mdk

In this paper we introduce a novel data format called \textbf{Homoiconic C}
(\textquotedblleft{}HC\textquotedblright{}) that we claim expresses a safe superset of C-like computations,
yet is even easier to read than existing scripting languages. The key is
a novel data structure and computational model we call \emph{Frames}, designed
to replace and improve upon the s-expressions and metacircular evaluation
model of Lisp{}[\mdcite{lisp}{4}]. Frames act as a universal monad\mdfootnote{2}{\noindent Not just the mathematical term for an object closed under all
operations, but Liebniz's original idea of a single universal
building-block{}[\mdcite{liebniz}{3}].%mdk
\label{fn-monad}%mdk%mdk
}, with the
result that Homoiconic C is simply conventions for creating and
evaluating Frames.%mdk

In this paper we will describe the complete syntax of HC, including three
types of lists, seven atoms, and the seven predefined operators; as a
modeless data format, there are no keywords, reserved words, or even any
grammar. Next we discuss the robust data protection semantics inspired by
BitC{}[\mdcite{shapiro:effecttyping}{8}], which allow us the efficiency of C's memory
model without any of the downsides. Finally, we provide examples of how
these deceptively simple primitives allow us to trivially implement HTML,
Object-Oriented programming, and other higher-level abstractions.%mdk

We are currently developing the first implementation of HC as a
TypeScript{}[\mdcite{typescript}{9}] interpreter. It can be found at
\href{http://github.com/TheSwanFactory/hclang/}{{\ttfamily http://\hspace{0pt}github.\hspace{0pt}com/\hspace{0pt}TheSwanFactory/\hspace{0pt}hclang/\hspace{0pt}}} under an MIT Open Source
license.%mdk
%mdk
\end{abstract}%mdk

\section{1.\hspace*{0.5em}Introduction}\label{sec-intro}%mdk%mdk

\noindent Blah.%mdk

\section{2.\hspace*{0.5em}Syntax}\label{sec-frames}%mdk%mdk

\noindent The core principle of Homoiconic C syntax is \emph{isomorphism}: every
syntactic construct refers to exactly one semantic concept. Put another
way, it is modeless : characters don't mean different things in different
places. This enables vastly simpler parsers and a much shallower learning
curve.%mdk

The syntax of traditional data formats is build around \emph{literals}, and
the \emph{terminals} that separate and organize them. HC simply extends that
with \emph{identifiers}, and a single grammar construct, the \emph{expression}. The
entire syntax is in TableSyntax.%mdk

\begin{table}[tbp]%mdk
\begin{mdcenter}%mdk
\begin{mdtabular}{3}{\dimeval{(\linewidth)/3}}{1ex}%mdk
\begin{tabular}{lll}\midrule
\multicolumn{3}{|c|}{{\bfseries Literals}}\\
\multicolumn{1}{|c}{{\bfseries Bit}}&\multicolumn{1}{|c}{{\bfseries Number}}&\multicolumn{1}{|c|}{{\bfseries Quote}}\\

\midrule
\multicolumn{1}{|l}{\textless{}\textgreater{} \# all (true)}&\multicolumn{1}{|l}{0b010 \# binary}&\multicolumn{1}{|l|}{}\\
\multicolumn{1}{|l}{() \# nil (false)}&\multicolumn{1}{|l}{0o1777 \# octal}&\multicolumn{1}{|l|}{}\\
\multicolumn{1}{|l}{}&\multicolumn{1}{|l}{1234 \# decimal}&\multicolumn{1}{|l|}{}\\
\multicolumn{1}{|l}{}&\multicolumn{1}{|l}{0xbeef \# hexadecimal}&\multicolumn{1}{|l|}{}\\
\multicolumn{1}{|l}{}&\multicolumn{1}{|l}{0@Base64 \# Base64}&\multicolumn{1}{|l|}{}\\
\midrule
\multicolumn{1}{|l}{}&\multicolumn{1}{|l}{}&\multicolumn{1}{|l|}{}\\
\end{tabular}\end{mdtabular}

\mdhr{}%mdk

\noindent\mdcaption{\textbf{Table~\mdcaptionlabel{1}.}~\mdcaptiontext{}}%mdk
%mdk
\end{mdcenter}%mdk
\end{table}%mdk

\subsection{2.1.\hspace*{0.5em}Literals}\label{sec-literals}%mdk%mdk

\subsection{2.2.\hspace*{0.5em}Expressions}\label{sec-expressions}%mdk%mdk

\subsection{2.3.\hspace*{0.5em}Terminals}\label{sec-terminals}%mdk%mdk

\noindent Terminals delimit and group expressions. There are eleven terminals
organized into three categories:%mdk

\begin{itemize}[noitemsep,topsep=\mdcompacttopsep]%mdk

\item Whitespace: newline \textbar{} space \textbar{} tab%mdk

\item Separators:  data \textquoteleft{},\textquoteright{} \textbar{} metadata \textquoteleft{};\textquoteright{}%mdk

\item Aggregates:  group \textquoteleft{}(\textquoteright{} \textquoteleft{})\textquoteright{} \textbar{} array '[' ']' \textbar{} closure \textquoteleft{}\{\textquoteright{} \textquoteleft{}\}\textquoteright{}%mdk
%mdk
\end{itemize}%mdk

\subsection{2.4.\hspace*{0.5em}Identifiers}\label{sec-identifiers}%mdk%mdk

\subsection{2.5.\hspace*{0.5em}Concepts}\label{sec-concepts}%mdk%mdk

\subsubsection{2.5.1.\hspace*{0.5em}Function}\label{sec-function}%mdk%mdk

\subsubsection{2.5.2.\hspace*{0.5em}Array}\label{sec-array}%mdk%mdk

\subsubsection{2.5.3.\hspace*{0.5em}Dictionary}\label{sec-dictionary}%mdk%mdk

\subsubsection{2.5.4.\hspace*{0.5em}Scope}\label{sec-scope}%mdk%mdk

\subsection{2.6.\hspace*{0.5em}Syntax}\label{sec-syntax}%mdk%mdk

\subsection{2.7.\hspace*{0.5em}Operators}\label{sec-operators}%mdk%mdk

\section{3.\hspace*{0.5em}Data Protection}\label{sec-data-protection}%mdk%mdk

\subsection{3.1.\hspace*{0.5em}Encapsulation}\label{sec-encapsulation}%mdk%mdk

\subsection{3.2.\hspace*{0.5em}Effect Typing}\label{sec-effect-typing}%mdk%mdk

\section{4.\hspace*{0.5em}Applications}\label{sec-applications}%mdk%mdk

\subsection{4.1.\hspace*{0.5em}Object-Orientation}\label{sec-object-orientation}%mdk%mdk

\noindent Let's program some Javascript:%mdk
\begin{mdpre}%mdk
\noindent{\mdcolor{navy}function}~hello()~\{\\
~~{\mdcolor{navy}return}~{\mdcolor{maroon}"}{\mdcolor{maroon}hello~world!}{\mdcolor{maroon}"}\\
\}%mdk
\end{mdpre}
\subsection{4.2.\hspace*{0.5em}Web Technologies}\label{sec-web-technologies}%mdk%mdk

\section{5.\hspace*{0.5em}Next Steps}\label{sec-next-steps}%mdk%mdk

\subsection{5.1.\hspace*{0.5em}Implementation Status}\label{sec-implementation-status}%mdk%mdk

\subsection{5.2.\hspace*{0.5em}Future Directions}\label{sec-future-directions}%mdk%mdk

\subsection{5.3.\hspace*{0.5em}Potential Implications}\label{sec-potential-implications}%mdk%mdk

\section{6.\hspace*{0.5em}Related Work}\label{sec-related-work}%mdk%mdk

\noindent\textbf{Note}.
The syntax highlighting works in the PDF output too.%mdk%mdk

\section{7.\hspace*{0.5em}Conclusion}\label{sec-conclusion}%mdk%mdk

\noindent Really fun to write Markdown :-)%mdk

\mdsetrefname{References}%mdk
{\mdbibindent{0}%mdk
\begin{thebibliography}{10}%mdk
\label{sec-bibliography}%mdk

\bibitem{clang}\mdbibitemlabel{}Stub.
\newblock  \emph{Stub}.
\newblock  Stub, 1900a.\label{clang}%mdk%mdk

\bibitem{forth}\mdbibitemlabel{}Stub.
\newblock  \emph{Stub}.
\newblock  Stub, 1900b.\label{forth}%mdk%mdk

\bibitem{liebniz}\mdbibitemlabel{}Stub.
\newblock  Stub.
\newblock  \emph{Stub}, 1900c.\label{liebniz}%mdk%mdk

\bibitem{lisp}\mdbibitemlabel{}Stub.
\newblock  Stub.
\newblock  \emph{Stub}, 1900d.\label{lisp}%mdk%mdk

\bibitem{lua}\mdbibitemlabel{}Stub.
\newblock  Stub.
\newblock  \emph{Stub}, 1900e.\label{lua}%mdk%mdk

\bibitem{scheme}\mdbibitemlabel{}Stub.
\newblock  \emph{Stub}.
\newblock  Stub, 1900f.\label{scheme}%mdk%mdk

\bibitem{self}\mdbibitemlabel{}Stub.
\newblock  Stub.
\newblock  \emph{Stub}, 1900g.\label{self}%mdk%mdk

\bibitem{shapiro:effecttyping}\mdbibitemlabel{}Stub.
\newblock  Stub.
\newblock  \emph{Stub}, 1900h.\label{shapiro:effecttyping}%mdk%mdk

\bibitem{typescript}\mdbibitemlabel{}Stub.
\newblock  Stub.
\newblock  \emph{Stub}, 1900i.\label{typescript}%mdk%mdk

\bibitem{unixshell}\mdbibitemlabel{}Stub.
\newblock  \emph{Stub}.
\newblock  Stub, 1900j.\label{unixshell}%mdk%mdk
\par%mdk
\end{thebibliography}}%mdk%mdk

\begin{mdbmargintb}{4em}{}%mdk
\begin{mdflushright}%mdk
{\tiny Created with~\href{https://www.madoko.net}{Madoko.net}.}%mdk
\end{mdflushright}%mdk
\end{mdbmargintb}%mdk%mdk


\end{document}
